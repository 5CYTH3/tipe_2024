\documentclass{beamer}
\usetheme{Berkeley}
\definecolor{UBCblue}{RGB}{221,160,221} % UBC Blue (primary)
\usecolortheme[named=UBCblue]{structure}

\usepackage{ebproof}
\usepackage{forest}
\usepackage{simplebnf}

%Information to be included in the title page:
\title{Inférence de types et Parsing dans la même passe}
\author{Enogad Le Biavant--Frederic}
\institute{Alain René Lesage MP2I}
\date{2024}

\begin{document}

\frame{\titlepage}

\section{Présentation générale}
\begin{frame}
\frametitle{Présentation générale}
Les principes ont l'air quand même assez similaires... \\
(defun f (x) + x 1)

\[
\begin{forest}
		[Func [f] [$x:\alpha$] [$Expr:\beta$ [+ [x] [1]]]]
\end{forest}
\]

\end{frame}

\subsection{Grammaire}
\begin{frame}
\frametitle{Grammaire}

\begin{bnf}
		$program$ ::= $expr$;;
		$expr$    ::= | $list$ | $atom$ | $function$;;
		$list$    ::= '(' [\{$expr$\}] ')';;
		$atom$    ::= | $id$ | $literal$;;
		$literal$ ::= | string | int | bool;;
		$function$::= '(' 'defun' $id$ '(' [\{$id$\}] ')' $expr$ ')';;
\end{bnf} \end{frame}
\subsection{HM Typesystem}
\begin{frame}
\frametitle{Hindley-Milner Typesystem}
\[
\begin{prooftree}
		\hypo{x:\sigma\in\Gamma}
		\infer1[var]{\Gamma \vdash x:\sigma}
\end{prooftree}
\]
\newline
\[
\begin{prooftree}
		\hypo{\Gamma, x:\tau \vdash e:\tau'}
		\infer1[abs]{\Gamma \vdash \lambda x.e:\tau \to \tau'}
\end{prooftree}
\]
\newline
\[
\begin{prooftree}
		\hypo{\Gamma \vdash f:\tau \to \tau'}
		\hypo{\Gamma \vdash e : \tau}
		\infer2[app]{\Gamma \vdash f\ e : \tau'}
\end{prooftree}
\]
\end{frame}

\subsection{RD Parser}
\begin{frame}
\frametitle{Parser Récursif Descendant}

\end{frame}

\section{Ma démonstration}
\begin{frame}
\frametitle{Ma démonstration}
\end{frame}

\section{Et après ?}
\begin{frame}
\frametitle{Et après ?}
\begin{itemize}
		\item<1-> Est-ce que il y a un réel impact sur la complexité temporelle ?
		\item<2->Pour quelles grammaires est-ce possible ?
		\item<3->Formalisation de cet outil ?
\end{itemize}
\end{frame}

\end{document}
